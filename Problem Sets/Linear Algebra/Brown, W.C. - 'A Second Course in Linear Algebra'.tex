\documentclass[letterpaper,11pt]{article}
\usepackage{tabularx} % extra features for tabular environment
\usepackage{amsmath}  % improve math presentation
\usepackage{graphicx} % takes care of graphic including machinery
\usepackage[margin=1in,letterpaper]{geometry} % decreases margins
\usepackage{cite} % takes care of citations
\usepackage[final]{hyperref} % adds hyper links inside the generated pdf file
\usepackage{datetime}
\usepackage{caption}
\usepackage{tikz, pgfplots}
\usepackage[bottom]{footmisc}
\usepackage{tikz}
\usepackage{multicol}
\usepackage{booktabs}
\usepackage{environ}
\usepackage{hyperref}
\usepackage{xcolor}
\usepackage{amsthm}
\usepackage{bm}
\usepackage{amssymb}
\usepackage{titlesec}
\usepackage{systeme}
\usepackage{enumitem}
\usepackage{graphics}
\usepackage{setspace}
\usepackage{tikz-cd}
\usepackage{adjustbox}
\usepackage{amsthm}
\usepackage{xpatch}
\usepackage{capt-of}
\usepackage[T1]{fontenc}
\usepackage{pdfcolparallel}
\usepackage{bussproofs}
%\usepackage{dutchcal}
\usepackage{eucal}
\DeclareMathAlphabet\mathrsfso      {U}{rsfso}{m}{n}

\newcommand{\lineb}{\vspace*{0.2cm}
\hrulefill
\vspace*{0.2cm}}

\renewcommand{\labelitemi}{\tiny$\bullet$}
% https://tex.stackexchange.com/questions/390521/vdash-and-models-with-curly-horizontal-lines
\newcommand{\vsim}{\mathrel{\scalebox{1}[1.5]{$\shortmid$}\mkern-3.1mu\raisebox{0.15ex}{$\sim$}}}


\usetikzlibrary{arrows.meta,automata,positioning,fit,shapes.geometric}

\theoremstyle{definition}
\newtheorem*{definition}{Definition:}%[section]

\newtheorem*{fact}{Fact:}
\newtheorem*{facts}{Facts:}
\newtheorem*{expl}{Example:}
\newtheorem*{corollary}{Corollary:}
\newtheorem*{lemma}{Lemma:}
\newtheorem*{theorem}{Theorem:}
\xpatchcmd{\@thm}{\thm@headpunct{.}}{\thm@headpunct{}}{}{}

\newtheorem*{remark}{Remark:}%[section]

% Citation: https://tex.stackexchange.com/questions/4216/how-to-typeset-correctly
\newcommand*{\defeq}{\mathrel{\vcenter{\baselineskip0.5ex \lineskiplimit0pt
                     \hbox{\scriptsize.}\hbox{\scriptsize.}}}%
                     =}
\newcommand*{\defqe}{=\mathrel{\vcenter{\baselineskip0.5ex \lineskiplimit0pt
                     \hbox{\scriptsize.}\hbox{\scriptsize.}}}%
                     }
\makeatletter
\AtBeginDocument{\xpatchcmd{\@thm}{\thm@headpunct{.}}{\thm@headpunct{}}{}{}}
\makeatother

\begin{document}

\title{A Second Course in Linear Algebra \\ {\small Solution Reference: Jacob Noel\footnote{Unless otherwise cited, I derived the following answers as exercises for myself. }}}
\author{Brown, William C.}
%\date{\today}


{\let\newpage\relax\maketitle}

% ========================================================= %
\pagebreak
\section*{Chapter I - Linear Algebra}

\subsection*{Section 1 - Definitions and Examples of Vector Spaces}


\begin{flushleft}
	1.
\end{flushleft}

We need to show that $(\mathbb{F}_p, \oplus, \cdot)$ is a field, where $p$ is a positive prime in $\mathbb{Z}$ and $\mathbb{F}_p = \{0, 1, \cdots, p - 1\}$. We define $x \oplus y = r$ to be such that $x + y = kp + r$ and $x \cdot y = w$ where $x \cdot y = k^\prime p + w$ and $x, y, r, w \in \mathbb{F}_p$ and $k, k^\prime \in \mathbb{Z}$. 

First, demonstrating that $\oplus$ and $\cdot$ are well defined\footnote{This appears to be assumed by the problem, but for the same of completeness this will be written out as well.}, we need to show that such a solution $x + y = kp + r$ is unique. Therefore, let $x + y = k_1p + r_1$ and $x + y = k_2p + r_2$, this gives \[k_1p + r_1 = k_2p + r_2\] which may be rewritten $(k_1 - k_2)p = r_2 - r_1$. This implies $p \mid r_2 - r_1$, but since $r_1, r_2 \in \mathbb{F}_p$ we have $0 \leq |r_2 - r_1| \leq p-1$ giving $r_2 - r_1 = 0$ and $r_1 = r_2$ and therefore such a solution is unique. Likewise the argument for $\cdot$ is identical. 

Now, validating our field axioms, for the following let $x, y, z, r_i, w_i \in \mathbb{F}_p$ and $k, k^\prime \in \mathbb{Z}$. 

\begin{itemize}[noitemsep]
	\item [V1.] We have $x \oplus y = r_1$ such that $x + y = k_1p + r_1$ and $y \oplus x = r_2$ such that $y + x = k_2p + r_2$, but $x + y = y + x$ in $\mathbb{Z}$, so $k_1p + r_1 = k_2p + r_2$ and equality follows from the uniqueness shown above.
	\item [V2.] TODO
	\item [V3.] Note that $x \oplus 0 = r$ in $x + 0 = x = pk + r$ but $0 \leq x \leq p - 1$ therefore $x = r$, and hence $0 \in \mathbb{F}_p$ is an identity for $\oplus$. 
	\item [V4.] TODO
	\item [V5.] TODO
	\item [V6.] TODO
	\item [V7.] TODO
	\item [V8.] TODO
	\item [V9.] TODO
\end{itemize}

\hrulefill

\pagebreak
\begin{flushleft}
	2.
\end{flushleft}

We seek to verify that the field of rational functions over $\mathbb{R}$, given by $$\mathbb{R}(X) = \left\{ \frac{f(x)}{g(x)} \mid f, g \in \mathbb{R}[X] \text{ and } g\neq 0 \right\}$$

is a field under, $$\frac{f}{g} + \frac{h}{k} = \frac{kf+gh}{gk} \text{ and } \frac{f}{g} \cdot \frac{h}{k} = \frac{fh}{gk}$$

As before, verifying that our operations are well-founded, let TODO


\hrulefill

\pagebreak
\begin{flushleft}
	3.
\end{flushleft}

\begin{lemma}[Subfield Criterion]
	A nonempty subset of a field $S \subseteq F$ is a subfield under the same operations of $F$ if it is closed under the operations of $F$, and taking additive and multiplicative inverses (when $x$ is nonzero).
\end{lemma}

\textit{Proof.}

Our field axioms F1, F2, F5, F6, F9 for $S$ are immediate consequences of the fact that the operations of $S$ are borrowed from the field $F$, while F4 and F8 are part of our assumption. This leaves only F3 and F7. 

Note though, if $S$ is closed, nonempty, and contains additive inverses, then there exists $a, -a \in S$, implying $a - a = 0 \in S$ by closure, likewise for multiplication.

\begin{flushright}
	$\square$
\end{flushright}



\hrulefill 
\vspace*{2mm}


We seek to show that $F = \{a + b\sqrt{-5} \mid a, b \in \mathbb{Q}\} \subseteq \mathbb{C}$ is a subfield of $\mathbb{C}$. First, it is simple to note that the set is a subset of $\mathbb{C}$ and since our operations are borrowed from $\mathbb{C}$ we may apply the above lemma. Therefore we need only verify that $F$ is nonempty, closed under addition, multiplication, and taking of additive and multiplicative inverses.

First, $F$ is clearly nonempty since $a = b = 0 \in \mathbb{Q}$ implies $0 \in F$.

Suppose $x = a+b\sqrt{-5}, y = c + d\sqrt{-5} \in F$, then $x + y = (a + c) + (b + d)\sqrt{-5}$, and since $a + c, b + d \in \mathbb{Q}$ by additive closure of $\mathbb{Q}$, we have $x + y \in F$. Similarly, note that $$(a + b\sqrt{-5}) (c + d\sqrt{-5}) = (ac -5bd) + (ad + bc)\sqrt{-5}$$

and by multiplicative closure of $\mathbb{Q}$ we have $xy \in F$.

If $x$ defined as above is in $F$, then $-x = -a - b\sqrt{-5} \in F$, since $-a, -b \in \mathbb{Q}$ by existence of additive inverses of $\mathbb{Q}$. 

Lastly, $$\frac{1}{x} = \frac{1}{a + b\sqrt{-5}} = \frac{1}{a + b\sqrt{-5}}\cdot\frac{a-b\sqrt{-5}}{a - b\sqrt{-5}} = \frac{a-b\sqrt{-5}}{a^2 -5b^2} = \left(\frac{a}{a^2-5b^2}\right) + \left(\frac{-b}{a^2-5b^2}\right)\sqrt{-5}$$

which is in $F$ by closure and inverses of $\mathbb{Q}$. 

\begin{flushright}
	$\square$
\end{flushright}

\hrulefill 
\vspace*{2mm}

To see that $F^\prime = \{a + b\sqrt{-5} \mid a, b \in \mathbb{Z}\}$ is not a subfield, consider that by the last equation from the previous proof, when $a = 2$ and $b = 0$ we have $2 + 0\sqrt{-5} = 2 \in F^\prime$ but $$\frac{1}{2} = \frac{2}{2^2} + \frac{0}{2^2 - 0}\sqrt{-5} = \frac{1}{2} \notin \mathbb{Z}$$






\pagebreak

\begin{flushleft}
	4.
\end{flushleft}

We seek to show that, $$V_a = \{f \in \mathbb{R} \mid \text{ $f$ has a derivative at $a$}\}$$ for $a \in I$ an open interval of $\mathbb{R}$, is a subspace of $\mathbb{R}^I$. 

\hrulefill

Since this requires real analysis, let us be cautious and lay out definitions. 
TODO

\begin{flushleft}
	5.
\end{flushleft}

The most obvious vector space over $\mathbb{R}^\mathbb{N}$ is simply elementwise addition and scalar multiplication. That is, 

$$\{x_i\} + \{y_i\} = \{x_i + y_i\} \text{ and } \alpha \{x_i\}= \{\alpha x_i\}$$

for $i \in \mathbb{N}$ and $x_i, y_i, \alpha \in \mathbb{R}$. 

Verifying this for our vector space axioms, 

\begin{itemize}[noitemsep]
	\item V1. $$\{x_i\} + \{y_i\} = \{x_i + y_i\} = \{y_i + x_i\} = \{y_i\} + \{x_i\}$$
	\item V2. $$\{x_i\} + \left(\{y_i\} + \{z_i\}\right) = \{x_i + y_i + z_i\} = (\{x_i\} + \{y_i\}) + \{z_i\} $$
	\item V3. Let $\textbf{0} = \{y_i\}$ for $y_i = 0$ for all $i \in \mathbb{N}$, then $$\{x_i\} + \textbf{0} = \{x_i\} + \{y_i\} = \{x_i + 0\} = \{x_i\}$$ 
	\item V4. Let\footnote{There may be some harmless abuse of notation here depending on how we precisely construct of these as sets.} $\textbf{x} = \{x_i\} \in \mathbb{R}^\mathbb{N}$, then $-\textbf{x} = \{-x_i\}$, giving $$\textbf{x} + (-\textbf{x}) = \{x_i\} + \{-x_i\} = \textbf{0}$$
	\item V5. $$(\alpha\beta) \{x_i\} = \{(\alpha \beta) x_i \} = \{\alpha (\beta x_i)\} = \alpha \{\beta x_i\} = \alpha (\beta \{x_i\})$$ for $\alpha, \beta, x_i \in \mathbb{R}$
	\item V6. $$\{x_i\}(\alpha + \beta) = \{x_i(\alpha + \beta)\} = \{x_i\alpha + x_i\beta\} = \{x_i\alpha\} + \{x_i\beta\} = \{x_i\}\alpha + \{x_i\}\beta$$
	\item V7. $$(\{x_i\} + \{y_i\})(\alpha)$$ TODO
\end{itemize}


\pagebreak
\begin{flushleft}
	6.
\end{flushleft}

\begin{lemma}[Subspace Criterion]
	A nonempty subset of a vector space $W \subseteq V$ is a subspace under the same operations of $V$ if it is closed under the vector addition and scalar multiplication of $V$.
\end{lemma}

\textit{Proof.}

Similarly to fields, we may note that V1, V2, V5, V6, V7, V8 all follow from the vector space axioms of $V$. Therefore the only remaining are V3 and V4, but closure of $W$ under scalar multiplication implies\footnote{This relies on the fact that the additive inverse of our multiplicative identity takes elements to their additive identity. To verify this, note that $1 \in V$, therefore there exists $-1 \in V$ such that $1 + (-1) = 0$, but then $$(1+ (-1))x = 1x + (-1)x = x + (-1)x = 0$$ Therefore $(-1)x$ must be the additive inverse of $x$.} that for $x \in W$ we have $(-1)x \in W$ so $x + (-1)x = 0 \in W$ and therefore both V3 and V4 are satisfied. 


\begin{flushright}
	$\square$
\end{flushright}


\begin{flushleft}
	(a.)
\end{flushleft}

Let $x, y \in W_1$, then $x = \{x_i\}, y = \{y_i\} \in \mathbb{R}^\mathbb{N}$ with, $$\lim_{i \to \infty} x_i = \lim_{i \to \infty} y_i = 0$$ 

then our limit addition laws imply, 

$$\lim_{i \to \infty} x_i + \lim_{i \to \infty} y_i = \lim_{i \to \infty} (x_i +  y_i) = 0$$
 
therefore $\{x_i + y_i\} \in W_1$ and hence $W_1$ is closed under addition. We have also that $\lim_{i \to \infty} x_i = 0$ implies $\alpha(\lim_{i \to \infty}x_i) = 0$ therefore $\alpha \{x_i\}  \in W_1$ and $W_1$ is therefore closed under scalar multiplication. Noting that $\textbf{0} \in W_1$ completes the proof that $W_1$ is a subspace of $\mathbb{R}^\mathbb{N}$. 

\begin{flushright}
	$\square$
\end{flushright}

\begin{flushleft}
	(b.)
\end{flushleft}

Let $x, y \in W_2$, then $\{x_i\}$ and $\{y_i\}$ are bounded sequences. Let $|x_i| \leq M_1$ for all $x_i$ and $|y_i| \leq M_2$ for all $y_i$. Then $|x_i + y_i| \leq M_1 + M_2$ and hence $\{x_i + y_i\}$ is bounded and therefore in $W_2$, therefore $W_2$ is closed under addition. Similarly noting that $|\alpha x_i| \leq |\alpha|M_1$ shows that $W_2$ is closed under scalar multiplication and since $\textbf{0} \in W_2$ all the conditions for $W_2$ to be a subspace are satisfied. 

\begin{flushright}
	$\square$
\end{flushright}

\begin{flushleft}
	(c.)
\end{flushleft}

As before $\textbf{0} \in W_3$ therefore $W_3$ is nonempty. Suppose $x, y \in W_3$, then $$\sum_{i=1}^\infty x_i^2 < \infty \text{ and }\sum_{i=1}^\infty y_i^2 < \infty$$

since $x_i, y_i \in \mathbb{R}$ and $0 \leq x_i^2, y_i^2$, $$\left\{\sum_{i=1}^n x_i^2\right\}_{n=1}^\infty \text{ and }\left\{\sum_{i=1}^n y_i^2\right\}_{n=1}^\infty$$ are increasing and are bounded (by assumption) therefore by monotone convergence each has limits $\ell_1$ and $\ell_2$. By limit arithmetic,

$$\lim_{n \to \infty} \sum_{i=1}^n x_i^2  + \lim_{n \to \infty} \sum_{i=1}^n y_i^2  = \lim_{n \to \infty}\left(\sum_{i=1}^n x_i^2  +  \sum_{i=1}^n y_i^2\right) = \lim_{n \to \infty}\left(\sum_{i=1}^n x_i^2  + y_i^2\right)$$

therefore, $$\sum_{i=1}^\infty (x_i^2 + y_i^2)$$




\pagebreak

\begin{flushleft}
	7.
\end{flushleft}

Let $(F^n)^* = F^n - \{0\}$



\hrulefill

\begin{flushleft}
	8.
\end{flushleft}
TODO
\hrulefill

\begin{flushleft}
	9.
\end{flushleft}

Let $W_1, W_2 \subseteq V$ be subspaces of $V$ such that $W_1 \cup W_2 = W^\prime$ is a subspace of $V$. Suppose $W_1 \nsubseteq W_2$ and $W_2 \nsubseteq W_1$, then there exists $a \in W_1$ such that $a \notin W_2$ and $b \in W_2$ such that $b \notin W_1$. Then we have that $a + b \in W^\prime$, but since $W^\prime = W_1 \cup W_2$ then $a + b \in W_1$ or in $W_2$. Without loss of generality suppose $a + b \in W_1$, then since $W_1$ is a vector space $(a + b) - a = b \in W_1$, a contradiction. 

\begin{flushright}
	$\square$
\end{flushright}


\hrulefill

\begin{flushleft}
	10.
\end{flushleft}

TODO


\hrulefill

\begin{flushleft}
	11.
\end{flushleft}

Begin by rewriting\footnote{This is just the substitution $L = A, B = K, M = X$ with some reordering by commutativity of $+$ and $\cap$, but I find the labeling slightly clearer to remember.} the modular law in the following way, $$A + (X \cap B) = (A + X) \cap B$$ with $A \subseteq B$, for subspaces $A, B, X$. 

Let $v \in A + (X \cap B)$ then we have $v = a + z$ for some $z \in X$ with $z \in B$. Then, since $A \subseteq B$, we have $a \in B$, therefore $a + z \in B$. Similarly, $z \in X$ therefore $a + z \in A + X$, therefore since $v \in A + X, B$ then $v \in (A + X) \cap B$ and $A + (X \cap B) \subseteq (A + X) \cap B$. 

Let $v \in (A + X) \cap B$, then $v \in A + X$ and $v \in B$. Since $v \in A + X$ we may write $v = a + x$, for $a \in A$ and $x \in X$ with $a + x \in B$. Note though, since $B$ is a vector space and $a \in B$, we have $(a + x) - a = x \in B$, therefore $x \in X \cap B$ and $a + x \in A + (X \cap B)$, giving $(A + X) \cap B \subseteq A + (X \cap B)$. 

\begin{flushright}
	$\square$
\end{flushright}


\hrulefill

\begin{flushleft}
	12.
\end{flushleft}
TODO
\hrulefill

\begin{flushleft}
	13.
\end{flushleft}
TODO
\hrulefill

\begin{flushleft}
	14.
\end{flushleft}
TODO
\hrulefill

\begin{flushleft}
	15.
\end{flushleft}
TODO
\hrulefill

\begin{flushleft}
	16.
\end{flushleft}
TODO
\hrulefill

\begin{flushleft}
	17.
\end{flushleft}
TODO
\hrulefill

\begin{flushleft}
	18.
\end{flushleft}

Let $M_1, M_2 \in M_{n \times n}(F)$ such that $M_i = M_i^T$. 

\hrulefill

\begin{flushleft}
	19.
\end{flushleft}
TODO
\hrulefill

\begin{flushleft}
	20.
\end{flushleft}




\subsection*{Section 2 - Bases and Dimension}
\subsection*{Section 3 - Linear Transformations}
\subsection*{Section 4 - Products and Direct Sums}
\subsection*{Section 5 - Quotient Spaces and the Isomorphism Theorems}
\subsection*{Section 6 - Duals and Adjoints}
\subsection*{Section 7 - Symmetric Bilinear Forms}



\pagebreak
\section*{Chapter II - Multilinear Algebra}

\pagebreak
\section*{Chapter III - Canonical Forms of Matrices}

\pagebreak
\section*{Chapter IV - Normed Linear Vector Spaces}

\pagebreak
\section*{Chapter V - Inner Product Spaces}



\end{document}
