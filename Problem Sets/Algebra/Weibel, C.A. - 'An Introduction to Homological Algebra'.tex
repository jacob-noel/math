\documentclass[letterpaper,12pt]{article}
\usepackage{tabularx} % extra features for tabular environment
\usepackage{amsmath}  % improve math presentation
\usepackage{graphicx} % takes care of graphic including machinery
\usepackage[margin=1in,letterpaper]{geometry} % decreases margins
\usepackage{cite} % takes care of citations
\usepackage[final]{hyperref} % adds hyper links inside the generated pdf file
\usepackage{datetime}
\usepackage{caption}
\usepackage{tikz, pgfplots}
\usepackage[bottom]{footmisc}
\usepackage{tikz}
\usepackage{multicol}
\usepackage{booktabs}
\usepackage{environ}
\usepackage{hyperref}
\usepackage{xcolor}
\usepackage{amsthm}
\usepackage{bm}
\usepackage{amssymb}
\usepackage{titlesec}
\usepackage{systeme}
\usepackage{enumitem}
\usepackage{graphics}
\usepackage{setspace}
\usepackage{tikz-cd}
\usepackage{adjustbox}
\usepackage{amsthm}
\usepackage{xpatch}
\usepackage{capt-of}
\usepackage[T1]{fontenc}
\usepackage{pdfcolparallel}
\usepackage{bussproofs}
%\usepackage{dutchcal}
\usepackage{eucal}
\DeclareMathAlphabet\mathrsfso      {U}{rsfso}{m}{n}

\newcommand{\lineb}{\vspace*{0.2cm}
\hrulefill
\vspace*{0.2cm}}

\renewcommand{\labelitemi}{\tiny$\bullet$}
% https://tex.stackexchange.com/questions/390521/vdash-and-models-with-curly-horizontal-lines
\newcommand{\vsim}{\mathrel{\scalebox{1}[1.5]{$\shortmid$}\mkern-3.1mu\raisebox{0.15ex}{$\sim$}}}


\usetikzlibrary{arrows.meta,automata,positioning,fit,shapes.geometric}

\theoremstyle{definition}
\newtheorem*{definition}{Definition:}%[section]

\newtheorem*{fact}{Fact:}
\newtheorem*{facts}{Facts:}
\newtheorem*{expl}{Example:}
\newtheorem*{corollary}{Corollary:}
\newtheorem*{lemma}{Lemma:}
\newtheorem*{theorem}{Theorem:}
\xpatchcmd{\@thm}{\thm@headpunct{.}}{\thm@headpunct{}}{}{}

\newtheorem*{remark}{Remark:}%[section]

% Citation: https://tex.stackexchange.com/questions/4216/how-to-typeset-correctly
\newcommand*{\defeq}{\mathrel{\vcenter{\baselineskip0.5ex \lineskiplimit0pt
                     \hbox{\scriptsize.}\hbox{\scriptsize.}}}%
                     =}
\newcommand*{\defqe}{=\mathrel{\vcenter{\baselineskip0.5ex \lineskiplimit0pt
                     \hbox{\scriptsize.}\hbox{\scriptsize.}}}%
                     }
\makeatletter
\AtBeginDocument{\xpatchcmd{\@thm}{\thm@headpunct{.}}{\thm@headpunct{}}{}{}}
\makeatother

\begin{document}

\title{Algebra: Chapter 0 \\ {\small Answer Reference: Jacob Noel\footnote{Unless otherwise cited, I derived the following answers as exercises for myself.}}}
\author{Aluffi, Paolo}
%\date{\today}


{\let\newpage\relax\maketitle}

% ========================================================= %

\pagebreak
\section*{Chapter I - Preliminaries: Set theory and categories}


\pagebreak

\section*{Chapter II - Groups, first encounter}

\subsection*{Section 1 - Definition of group}

\begin{flushleft}
	1.1
\end{flushleft}

\begin{quote}
	\textit{Write a careful proof that every group is the group of isomorphisms of a groupoid. In particular, every group is the group of automorphisms of some object in some category.}
\end{quote}

TODO


\subsection*{Section 2 - Examples of groups}



\subsection*{Section 3 - The category \textbf{Grp}}
\subsection*{Section 4 - Group homomorphisms}
\subsection*{Section 5 - Free groups}
\subsection*{Section 6 - Subgroups}
\subsection*{Section 7 - Quotient groups}
\subsection*{Section 8 - Canonical decomposition and Lagrange's theorem}
\subsection*{Section 9 - Group actions}
\subsection*{Section 10 - Group objects in categories}


\begin{flushleft}
	1.
\end{flushleft}





\end{document}
